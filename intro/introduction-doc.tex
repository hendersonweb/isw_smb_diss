Identity underpins virtually every aspect of your life today. Nowhere is trust more critical than in the realm of identity management (IdM), where the safeguarding of personal information and authentication of digital identities are paramount. As decentralized systems continue to proliferate, the need for robust mechanisms to establish and maintain trust within IdM frameworks becomes ever more pressing. Using online services, opening a bank account, voting in elections, buying property, securing employment. All these things require proving your identity. 
In traditional identity management, every service provider (or relying party) stores credentials of each user and enables them to authenticate directly to the business. However, this also means that the user needs to separately register and authenticate with each individual service they wants to use. Federated identity management simplifies this process for the user. Here, an identity provider or credential service provider acts as an intermediary manages user credentials and enables the user to register and sign on to various service providers(Kubach et al., 2020). So in both traditional and federated identity management users rely on centralized intermediaries like Certificate Authorities (CAs) or trusted third parties who issue, hold, and control your identifiers and attestations. The development of secure and federated digital identities in Europe over the past 20 years was driven forward by initiatives such as the "Large Scale Pilots" Stork (European Commission - DIGIT/A3, 2019) and Stork 2.0 funded by the European Commission. The results of these pilots formed an important basis for the electronic IDentification, Authentication, and trust Services (eIDAS) regulation (eIDAS, 2018). In Germany, the development of secure digital identities was primarily promoted by the government through the introduction of the electronic identity card (nPA) (Smart e-ID, 2021). Despite eIDAS and the nPA (neuer Personalausweis), the everyday and private sector use of digital identities by citizens continues to be dominated by username/password applications and the use of single-sign-on systems controlled by big international platform operators, who offer only lower levels of assurance.

After the adoption  of eIDAS, the 27 EU Member States signed a declaration creating the European Blockchain Partnership (EBP) with the ambition to provide digital public services matching the required level of digital security and maturity which evolved to the commencement of European Blockchain Services Infrastructure (EBSI) (European Blockchain Services Infrastructure (EBSI), 2019)in 2019. EBSI’s aim is to enhance cross-border public services provided to citizens and businesses, to enhance government or public authorities collaboration, in support of EU policies and in full compliance with EU regulation, meeting the highest standards in terms of sustainability, privacy, and security using Digital identities. After the successful adoption of the Verifiable Credential (VC) Data Model (Sporny et.al, 2022) and Decentralized Identifier (DID) (Sporny et. al, 2022) as standards in W3C, different pilot implementations were built to show how a decentralized digital identity can be built upon the new standard. The decentralized identity management approaches, however, follow a user-centric model of identity management as shown in Figure 1. This is supposed to address interoperability, security, and privacy concerns, given the privileged position of the identity provider. In this model, the user controls their identity data and interacts directly with the service providers without relying on a trusted intermediary. Verifiable claims of credentials that the user received from credential issuers are being shared by the user on a need-to-know basis. In decentralized identity ecosystems verifiable data registries for example Blockchain, Distributed Ledger Technologies (DLT), and Databases are mainly used as an integrity-protected “bulletin board” for a public key infrastructure (PKI) that supports the mapping of keys to identifiers [Le20]. In Germany, in late 2020  The Federal Ministry for Economic Affairs and Climate Action (BMWK) launched an innovative competition on “Showcase Secure Digital Identities” (SDIKA, 2023) to promote outstanding approaches for new open, interoperable, and easy-to-use ID ecosystems, which are to be tested in practice in model regions across different use cases in four large-scale projects. Following the developments in the digital identity space, the European Commission has also launched its Digital Identity regulation (European Commission, 2021) with recommendations in June 2021 which proposes a trusted and secure Digital Identity for all Europeans using a Digital Wallet. The European Digital Identity will be available to EU citizens, residents, and businesses who want to identify themselves or share personal/business information with service providers enabling users to control their data. Prior to rolling out wallets to its citizens and businesses EU Digital Identity Wallet is piloted in four large-scale projects (Digibyte, 2023), that launched on 1 April 2023. The objective of these projects is to test digital identity wallets in real-life scenarios spanning different sectors. Over 250 private companies and public authorities across 25 Member States and Norway, Iceland, and Ukraine will participate. All the different implementations discussed above will be built on decentralized identity principles and standards.


%\Blindtext
\section{Equations}
This is how to write equations
\begin{align}
	\bm{A}\bm{x} = \bm{b}
	\label{eq:Equation1}
\end{align}

\section{Forschungsfrage und Problembeschreibung}
\section{Beitrag dieser Arbeit}