%% command definitions 

% Anzeige, dass noch eine Zitation fehlt: \citationneeded
\newcommand{\citationneeded}{\textcolor[rgb]{1,0,0}{\small [citation needed]}}

% Markiere offene TODOs im text: \todo{Hier noch mehr.}.
% Hierfür gibt es auch Pakete
\newcommand{\todo}[1]{\textcolor[rgb]{1,0,0}{\small TODO: #1}}

% Abkürzungen im Text
\newcommand{\zB}{z.\,B. }
\newcommand{\uU}{u.\,U. }

% Bildquelle in Hellgrau, rechts: \imgssource{Getty Images}
\newcommand{\imgsource}[1]{\begin{flushright}
		\textcolor{black!40}{\tiny Source:~#1}	
\end{flushright}}

% Transponiert-Zeichen, z.B.: \( \bm p^\T \)
\newcommand{\T}{\mathrm{T}}

% matrix schnell setzen mit \bmat{a & b  \\ c & d}
\newcommand{\bmat}[1]{ \ensuremath{\begin{bmatrix} #1 \end{bmatrix}} }

%% mathe-Abkürzungen
\newcommand{\gdw}{\ensuremath{\Leftrightarrow}}
\newcommand{\folgt}{\Rightarrow}
\newcommand{\mustbe}{\ensuremath{\stackrel{!}{=}}}
\newcommand{\grad}{\ensuremath{\nabla}}
\newcommand{\R}{\ensuremath{\mathbb{R}}}
\newcommand{\partfrac}[2]{ \ensuremath{\frac{\partial #1}{\partial #2}} }
\newcommand{\ddiff}{\ensuremath{\mathrm{d}}}
\newcommand{\ddt}{\ensuremath{\frac{\ddiff}{\ddiff t}}}
\newcommand{\equates}{\ensuremath{\widehat{=}}}
\newcommand{\rme}{\ensuremath{\mathrm{e}}}
\newcommand{\Lscript}{\ensuremath{\mathcal{L}}}

% := (\eqdef) und =: (\defeq) für Definitionen
\newcommand*{\defeq}{\mathrel{\vcenter{\baselineskip0.5ex \lineskiplimit0pt
			\hbox{\scriptsize.}\hbox{\scriptsize.}}} =}
\newcommand*{\eqdef}{=\mathrel{\vcenter{\baselineskip0.5ex \lineskiplimit0pt
			\hbox{\scriptsize.}\hbox{\scriptsize.}}}}
\newcommand{\mi}{\mathrm{i}} % i für Imaginärzahlen

%% \dddot{} für dreifache Zeitableitung [https://tex.stackexchange.com/questions/131587/mathtools-dddot-cause-the-term-to-rise]
\newcommand\scaleddot{\scalebox{.89}{.}}
\makeatletter
\renewcommand{\dddot}[1]{%
	{\mathop{\kern\z@#1}\limits^{\makebox[0pt][c]{\vbox to-2.2\ex@{\kern-\tw@\ex@
					\hbox{\normalfont\scaleddot\kern-0.5pt\scaleddot\kern-0.5pt\scaleddot}\vss}}}}}
\renewcommand{\ddddot}[1]{%
	{\mathop{\kern\z@#1}\limits^{\makebox[0pt][c]{\vbox to-2.2\ex@{\kern-\tw@\ex@
					\hbox{\normalfont\scaleddot\kern-0.5pt\scaleddot\kern-0.5pt\scaleddot\kern-0.5pt\scaleddot}\vss}}}}}
\makeatother

\newcommand{\LTIsysSub}[1]{\ensuremath{\Sigma_{#1}: \bm{\dot x}_{#1}  = \bm A_{#1} \bm x_{#1} + \bm b_{#1} \bm u_{#1}, \bm y_{#1} = \bm C_{#1} \bm x_{#1}, \bm x_{#1,0} = \bm x_{#1}(t=t_0)}}

